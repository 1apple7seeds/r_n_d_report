
% LaTeX2e Style for MAS R&D and master thesis reports
% Author: Argentina Ortega Sainz, Hochschule Bonn-Rhein-Sieg, Germany
% Please feel free to send issues, suggestions or pull requests to:
% https://github.com/mas-group/project-report
% Based on the template created by Ronni Hartanto in 2003
%

%\documentclass[thesis]{mas_report}
\documentclass[rnd]{mas_report}

\usepackage{float}
\usepackage{url}
\usepackage{amsmath}
\usepackage{pdfpages}
\newcommand\norm[1]{\left\lVert#1\right\rVert}


% ****************************************************
% THIS INFORMATION SHOULD BE UPDATED FOR YOUR REPORT
% ****************************************************
\author{Abhishek Padalkar}	
\title{Dynamic Motion Primitives}
\supervisors{%
Prof. Dr. Paul Pl{\"o}ger\\
Alex Mitrevski
}
\date{15 August 2018}


% \thirdpartylogo{path/to/your/image}

\begin{document}
\begin{titlepage}	
    \maketitle
\end{titlepage}

%----------------------------------------------------------------------------------------
%	PREFACE
%----------------------------------------------------------------------------------------

\pagestyle{plain}


\cleardoublepage
\statementpage

\begin{abstract}
To be able to plan and execute the motion is one of the primary requirements of an autonomous robot to accomplish a given task. Over the time, numerous solutions were presented for motion planning, which are good enough for solving the problem of motion planning for practical problems. In this work, we present a learning from demonstration framework, based on the dynamic motion primitives, which allows us to program the robot by visual demonstration of the motion. Biologically inspired dynamic motion primitives learn the control policy behind the demonstrated motion in terms of attractor dynamics of a non-linear second order differential equation. We evaluate the performance of dynamic motion primitives for generalization of numerous demonstrated trajectories on two five degrees of freedom manipulators. For boosting the manipulation capabilities of these manipulators, we propose the idea of using the mobile base motion along with the manipulator motion which lead to whole body motion control framework for manipulation. Proposed whole body motion control framework was integrated in current software solution for manipulation of Toyota HSR robot.   
\end{abstract}



\tableofcontents
\listoffigures
\listoftables

%-------------------------------------------------------------------------------
%	CONTENT CHAPTERS
%-------------------------------------------------------------------------------

\mainmatter % Begin numeric (1,2,3...) page numbering

\pagestyle{mainmatter}

\subfile{chapters/ch01_introduction}
\subfile{chapters/ch02_stateoftheart}
\subfile{chapters/ch03_methodology}
\subfile{chapters/ch04_solution}
\subfile{chapters/ch05_evaluation}

\subfile{chapters/ch07_conclusion}


%-------------------------------------------------------------------------------
%	APPENDIX
%-------------------------------------------------------------------------------

\begin{appendices}
\subfile{chapters/appendix}

\end{appendices}

\backmatter
\nocite{*}
%-------------------------------------------------------------------------------
%	BIBLIOGRAPHY
%-------------------------------------------------------------------------------
\addcontentsline{toc}{chapter}{References}
\bibliographystyle{plain} % Use the plainnat bibliography style
\bibliography{bibliography} % Use the bibliography.bib file as the source of references

\end{document}
